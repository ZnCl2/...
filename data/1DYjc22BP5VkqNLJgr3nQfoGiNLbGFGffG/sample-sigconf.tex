\documentclass[sigconf]{acmart}

\usepackage{booktabs} % For formal tables
\usepackage{listings}


% Copyright
%\setcopyright{none}
%\setcopyright{acmcopyright}
%\setcopyright{acmlicensed}
\setcopyright{rightsretained}
%\setcopyright{usgov}
%\setcopyright{usgovmixed}
%\setcopyright{cagov}
%\setcopyright{cagovmixed}


% DOI
\acmDOI{10.475/123_4}

% ISBN
%\acmISBN{123-4567-24-567/08/06}

%Conference
\acmConference[WOODSTOCK'97]{ACM Woodstock conference}{July 1997}{El
  Paso, Texas USA} 
\acmYear{1997}
\copyrightyear{2016}

\acmPrice{15.00}


\begin{document}
\title{SIG Proceedings Paper in LaTeX Format}
\titlenote{Produces the permission block, and
  copyright information}
\subtitle{Extended Abstract}
\subtitlenote{The full version of the author's guide is available as
  \texttt{acmart.pdf} document}



\author{Htaf (Logan) Dwes}
%\authornote{}
%\orcid{1234-5678-9012}
\affiliation{%
  \institution{LiberIT Liberty Information Technology Services}
  %\streetaddress{P.O. Box 1212}
  \city{Owen Sound} 
  \state{Ontario} 
  \postcode{N4K 4R1}
}
\email{logan@liberit.ca}
% The default list of authors is too long for headers}
\renewcommand{\shortauthors}{B. Trovato et al.}


\begin{abstract}
%motivation
Motivated by the desire to maximize use of cheap GPU processing power by
increasing the kinds of algorithms that can profit from it.
% problem statement
Currently only data parallel algorithms can profit from running their code on 
GPU.\@
% approach
One way is by making a programming language whose code can be treated as data 
and streamed through an OpenCL kernel. 
% results
Herein is the design of a programming language that allows using code as data
that can be streamed through an OpenCL kernel. 
% conclusion
This way not only data parallel
algorithms can reap the benefits of GPU, but also any referentially transparent
code can. 
\end{abstract}

\begin{abstract}
This paper provides a sample of a \LaTeX\ document which conforms,
somewhat loosely, to the formatting guidelines for
ACM SIG Proceedings\footnote{This is an abstract footnote}. 
\end{abstract}

%
% The code below should be generated by the tool at
% http://dl.acm.org/ccs.cfm
% Please copy and paste the code instead of the example below. 
%
\begin{CCSXML}
<ccs2012>
 <concept>
  <concept_id>10010520.10010553.10010562</concept_id>
  <concept_desc>Computer systems organization~Embedded systems</concept_desc>
  <concept_significance>500</concept_significance>
 </concept>
 <concept>
  <concept_id>10010520.10010575.10010755</concept_id>
  <concept_desc>Computer systems organization~Redundancy</concept_desc>
  <concept_significance>300</concept_significance>
 </concept>
 <concept>
  <concept_id>10010520.10010553.10010554</concept_id>
  <concept_desc>Computer systems organization~Robotics</concept_desc>
  <concept_significance>100</concept_significance>
 </concept>
 <concept>
  <concept_id>10003033.10003083.10003095</concept_id>
  <concept_desc>Networks~Network reliability</concept_desc>
  <concept_significance>100</concept_significance>
 </concept>
</ccs2012>  
\end{CCSXML}

\ccsdesc[500]{Computer systems organization~Embedded systems}
\ccsdesc[300]{Computer systems organization~Redundancy}
\ccsdesc{Computer systems organization~Robotics}
\ccsdesc[100]{Networks~Network reliability}

% We no longer use \terms command
%\terms{Theory}

\keywords{ACM proceedings, \LaTeX, text tagging}


%\maketitle{}

%\input{samplebody-conf}
\include{vmOnOpenCL}

\bibliographystyle{ACM-Reference-Format}
\bibliography{sigproc} 

\end{document}

\documentclass[10pt, sigplan, preprint]{acmart}

\usepackage{booktabs} % For formal tables
\usepackage{bytefield}
\usepackage{hyperref}
\usepackage{tabulary}
\usepackage{listings}
\usepackage{booktabs} % For formal tables
\usepackage{tikz}
\usetikzlibrary{arrows,shapes,automata,petri,positioning}
\pagenumbering{arabic}
\settopmatter{printfolios=true}
\usepackage{fontspec}
\newfontfamily\uni{DejaVu Sans}

\tikzset{place/.style={circle,
        thick,
        draw=blue!75,
        fill=blue!20,
        minimum size=6mm,
    },
    transitionH/.style={rectangle,
        thick,
        fill=black,
        minimum width=8mm,
        inner ysep=2pt
    },
    transitionV/.style={rectangle,
        thick,
        fill=black,
        minimum height=8mm,
        inner xsep=2pt
    }
}


% Copyright
%\setcopyright{none}
%\setcopyright{acmcopyright}
%\setcopyright{acmlicensed}
%\setcopyright{rightsretained}
%\setcopyright{usgov}
%\setcopyright{usgovmixed}
%\setcopyright{cagov}
%\setcopyright{cagovmixed}


% DOI
%\acmDOI{10.475/123_4}

% ISBN
%\acmISBN{123-4567-24-567/08/06}

%Conference
\acmConference[SLE2017]{SLE Vancouver}{October 2017}{Vancouver, British
Columbia,
Canada} 
\acmYear{2017}
\copyrightyear{2017}

%\acmPrice{15.00}


\begin{document}
\title[]{Pyash: One Language to Unite Them All}
\titlenote{A vertically integrated software language based on the common 
features of the majority of human languages.}
%\subtitle{Extended Abstract}
%\subtitlenote{The full version of the author's guide is available as
%  \texttt{acmart.pdf} document}


\author{Htaf (Logan) Dwes}
%\authornote{}
\affiliation{%
  \institution{LiberIT Liberty Information Technology Services}
}
\email{logan@liberit.ca}

\renewcommand{\shortauthors}{Logan Dwes}


\begin{abstract}%motivation
    By using the fundamentals of human language, we may be able to achieve
    complete vertical integration for software languages, allowing one language
    to do everything from low level programming to chatting with humans. 

% problem statement
 % Software languages have only been around for a few decades, but due to the
 % limitations of their grammar and vocabulary there has been an explosion of 
 % specialized languages.  At the same time 
  Most
  software languages can't be used for making content, documentation or having a
  discussion. Because the vocabulary and grammar of most software languages is
  so limited, it is impossible to gain conversational fluency, thus they never
  rise to the status of human language, but sit as merely a code. 

% approach
  %By making a language based on the most
  %common features and words of human languages, software languages may become 
  %usable for human discourse and fluency, as well it may become extensible enough
  %that all the various software language purposes could be folded into a single
  %unifying language. 

% results
  After years of research, analysis, data-mining and prototypes, 
  a language has been made that not only allows for human-computer discourse and
  programming, but surprisingly can also be usable as a highly formal 
  pivot language between the majority of human languages. As of this
  writing, beta-testing seems to be just a short time away. 

% conclusion
  In conclusion, a single language to unite all languages is viable at least to
  the degree that it has already been implemented. 
  It can translate between controlled variants of most, possibly all human
  languages. Though further work is needed to prove that those controlled 
  variants can be easily learned by natural language speakers.  It can't 
  translate between different software languages, but may be able to vertically
  integrate all the purposes of software and human languages into one language.
  Further work is needed in order to prove that to a higher degree of certainty.
\end{abstract}

% The code below should be generated by the tool at
% http://dl.acm.org/ccs.cfm
% Please copy and paste the code instead of the example below. 
%

 \begin{CCSXML}
<ccs2012>
<concept>
<concept_id>10011007.10011006.10011008</concept_id>
<concept_desc>Software and its engineering~General programming languages</concept_desc>
<concept_significance>500</concept_significance>
</concept>
<concept>
<concept_id>10003120.10003121.10003124.10010870</concept_id>
<concept_desc>Human-centered computing~Natural language interfaces</concept_desc>
<concept_significance>300</concept_significance>
</concept>
</ccs2012>
\end{CCSXML}

\ccsdesc[500]{Software and its engineering~General programming languages}
\ccsdesc[300]{Human-centered computing~Natural language interfaces}

% We no longer use \terms command
%\terms{Theory}

\keywords{grammar, programming language, pivot language}

\maketitle{}
\section{Introduction}
%  Human language has been evolving for over a hundred thousand years, 
%  to the point that humans only need to use a single language to accomplish 
%  the vast majority of the tasks in their daily lives. 
%% problem statement
%  Software languages have only been around for a few decades, but due to the
%  limitations of their grammar and vocabulary there has been an explosion of 
%  languages each for doing their own little thing: programming, markup, 
%  modeling, ontologies and numerous domain specific ones.  At the same time 
%  there are still many important things outside the realm of software languages,
%  such as documentation, conversation, and laws.
%  Because contemporary software language grammar and vocabulary is so limited, 
%  it is nearly impossible for humans to hold a conversation or think fluently 
%  in a software language. 

The purpose of software languages is to help humans communicate with machines. 
To achieve this, the language has to be regular and sufficiently well defined
that both parties understand what the other is saying. Though contemporarily the
computer is programmed in one language, and error messages have a different
protocol (mini-language). 

Homo-sapien language has been evolving since at least Mitochondrial Eve, she lived
possibly one or two hundred thousand years ago. There are already over 6,000 human languages, so it
is acceptable to add another one, which happens to also be a general purpose
software language. 

The goal of the software language, is complete vertical
integration. So that if one is to reincarnate into a robot, then
everything from the lowest to highest levels can be accomplished using the 
same language.  Similar to how English can be used to
communicate everything from the lowest to highest levels. 

Several versions of implementing this idea have been made over the last decade. This paper
uses Pyash as the word for the pivot language (Intermediary Representation), 
it means language and is the result of data mining world language vocabularies
(\ref{vocabulary}).
 
Natural English is not formal enough to be directly used as a software
language, however Pyash English is.  Pyash also acts as a bridge, for 
high precision translation, so Pyash English documents could be rapidly and 
precisely translated to Pyash Hindi, Pyash Spanish, Pyash Swahili, or any 
other supported human language. 

This high precision translation could open the door to software languages for 
the majority of humanity which is not fluent in English. 

\section{Literature Review}
This section will review some of the common
references and mention how Pyash is different from them. 

The main difference is that Pyash is based on the fundamentals of human
language, and has a complete and orthogonal vocabulary.

\subsection{COBOL}
COBOL, originally intended to be a business programming language, 
was designed by several committees, some of the committee members were
unfamiliar with computer programming and-or linguistics, the committees also had
 issues with discontinuity of personnel. This led to a language that was neither
 very good for computer programming, nor very easy for humans to understand,
 while also having issues with repeals due to changing personnel. 

By contrast, Pyash's design takes into consideration many academic and real world sources for
its grammar (\ref{grammar}), vocabulary (\ref{vocabulary}) and instruction set
architecture (\ref{ISA}). It's evolution has also run through several different
iterations though all with the same informed personnel to keep it on track. 

\subsection{Hypertalk}
Hypertalk uses English keywords to replace common programming syntax symbols,
so is largely just a relexification of standard (ALGOL inspired) programming. 

Pyash on the other hand starts with a human grammar base and then adapts it to
usage as a software language. 

% \subsection{Inform7}
% Inform7 is a domain  specific language based on English. 
% It is only very useful for its domain of making adventure games. 
% 
% Like most other English inspired programming languages, 
% it is based on idiomatic usage that is not extensible. 
% 
% Pyash is general purpose, and has one unifying grammar. 

\subsection{Lojban}
Lojban is a language intended for human use, but based on the structure of
programming languages, in particular predicate logic\cite{LojbanIntro}.
Because of this Lojban is more of an API rather than a human language, making it
very difficult to gain fluency\cite{LojbanProblem}\cite{okrent2010in}.

While there have been some cursory motions towards making Lojban a programming
language\cite{LojbanPlus}, none have gotten much past the concept stage. 

The net result is that Lojban has not proven suitable for human communication, 
nor as a software language.  Though it has been a useful stepping stone and
point of inspiration.  

\section{Method}
%% approach
%  Linguistic Universals are thousands of features which are found to be common 
%  to large groups of human languages. By making a language based on the most
%  common features and words of human languages, software languages may become 
%  usable for human discourse and fluency. As well as becoming extensible enough
%  that all the various software language purposes could be folded into a single
%  unifying language. 

Many different approaches have been taken to the creation of software languages. 
Rather than basing Pyash on the Chomsky Hierarchy of formal languages and formal
grammars, it is based it on human grammar. 

\subsection{Grammar}\label{grammar}

Linguistic Universals\cite{LingUA} are patterns found systematically across
large groups of languages, possibly all languages. In particular all languages
have verb phrases and noun phrases, and mark their phrases either with
placement, adpositions or affixes. All can also express tense, mood and aspect. 

However there is the issue of making the pivot language. Which of
the many options should the language use? To the rescue comes the World Atlas of Language 
Structures\cite{WALS} (WALS), which allows one to see what are the most common features
around the world. 

In particular Pyash is Verb-final, or Subject-Object-Verb
word-order, similar to Hindi, Japanese and Amharic.
%(Indo-Aryan languages), Turkish (Turkic languages),
% Tamil (Dravidian languages), Georgian (Kartvelian languages), Dani (Papuan
% languages), most East-African, Siberian and Australian languages. 
 Linguistic Universals point toward
suffixes and-or postpositions for verb-final languages, so they are used.

But what of the grammar words themselves? A variety of
contenders were reviewed, such as Universal Networking Language\cite{UNL} from the United
Nations University, and FrameNet\cite{FrameNet} from Berkley.
A more organic solution was chosen consisting of the list of Glossing 
Abbreviations\cite{glossAbv} used by linguists when transcribing foreign 
languages. 

\subsection{Vocabulary}\label{vocabulary}

Contemporary Software Languages generally lack a root vocabulary. Keywords may 
have a special meaning, but they are typically of a syntactic or grammatical
nature, so are at most a grammatical vocabulary. 
API's naming convention of being series of unreserved letters, means that all
unreserved words are proper nouns.  

Pyash has a root vocabulary so that documentation, description and discussion can 
all happen in the same language as computer programming. The encoding requires
API names to be  words with a proper morphology (\ref{morphology}), and may be
restricted to only being official dictionary defined ones, ensuring 
standardization and ease of translation.  

To generate the vocabulary first several word-lists were put together, including
  WordNet core\cite{wordnet}, 
  Oxford-3000\cite{oxford3000},
  UNL-core\cite{UNL},
Special English\cite{SpecialEnglish},
FrameNet\cite{FrameNet},
New Academic Word List (NAWL)\cite{NAWL}, 
New General Service List (NGSL)\cite{NGSL} and
Project Gutenberg Frequency List\cite{GutenbergFL}.  After collating them all
and taking out the duplicates, the language was left with almost 39 thousand words.

Google Cloud Translation API\cite{googleTranslate} was used
to translate each word on the list individually into the top 48 languages by 
number of native speakers. Giving an overall coverage of greater than 70\% of
the world population.

A script to sort the vocabulary based on the frequency
list\cite{GutenbergFL} was made and it filtered them for uniqueness. 
Words were removed that were:

\begin{description}
  \item[Overborrowed] If more than $38\%$\footnote{$2-\phi = 38\%$ where $\phi$
    is golden ratio or 1.618. A golden fraction was felt to be a natural choice.} 
    languages use the English term. 
  \item[Ambigious] If it means multiple things in more than $38\%$ of the
    languages. 
  \item[Homographs] If it is a homograph of an already defined word in any of the
    languages. 
\end{description}

This left the language with a fairly orthogonal pool of about eight thousand words. 

\subsection{Morphology}\label{morphology}
The pivot language needs to be sufficiently easily spoken by humans 
for it to be usable by humans in conversation. This was particularly the case
in early prototypes, as it wasn't realized that the pivot language could be
used for translating between possibly all human languages --- which would negate
the need for actually learning the pivot language, a Pyash controlled 
natural language would be sufficient. 

\begin{table}
  \begin{tabular}{llll}
ASCII&IPA&Description&English\\
    \midrule{}
a & {\uni{} ä} & central open vowel &\underline{a}rm\\
b & {\uni{} b} & voiced bilabial plosive &\underline{b}all\\
c & {\uni{} ʃ} & unvoiced post-alveolar fricative & \underline{sh}out\\
d & {\uni{} d} & voiced alveolar dental&\underline{d}oor\\
e & {\uni{} e̞} & mid front unrounded vowel&\underline{e}nter\\
f & {\uni{} f} & unvoiced labio dental fricative&\underline{f}ire\\
g & {\uni{} g} & voiced velar plosive&\underline{g}reat\\
h & {\uni{} ʰ} & aspiration&\underline{h}appy\\
i & {\uni{} i} & unrounded closed front vowel&sk\underline{i}\\
j & {\uni{} ʒ} & voiced post-alveolar fricative&gara\underline{g}e\\
k & {\uni{} k} & unvoiced velar plosive&\underline{k}eep\\
l & {\uni{} l} & lateral approximants&\underline{l}ove\\
m & {\uni{} m} & bilabial nasal&\underline{m}ap\\
n & {\uni{} n} & alveolar nasal&\underline{n}ap\\
o & {\uni{} o̞} & mid back rounded vowel&r\underline{o}bot\\
p & {\uni{} p} & unvoiced bilabial plosive&\underline{p}an\\
q & {\uni{} ŋ} & velar nasal&E\underline{ng}lish\\
r & {\uni{} r} & alveolar trill& (Scottish) cu\underline{r}d\\
s & {\uni{} s} & unvoiced alveolar fricative & \underline{s}nake\\
t & {\uni{} t} & unvoiced alveolar plosive & \underline{t}ime \\
u & {\uni{} u} & rounded closed back vowel & bl\underline{ue}\\
v & {\uni{} v} & voiced labio dental fricative & \underline{v}oice\\
w & {\uni{} w} & labio velar approximant & \underline{w}ater\\
x & {\uni{} x} & velar fricative& (Scottish) lo\underline{ch}\\
y & {\uni{} j} & palatal approximant & \underline{y}ou\\
z & {\uni{} z} & voiced alveolar fricative & \underline{z}oom\\
\@. & {\uni{} ʔ} & glottal stop & uh\underline{-}oh\\
6 & {\uni{} ə} & mid central vowel&\underline{uh}\\
7 & {\uni{} ˦} & high tone& wha\underline{?}\\
\_ & {\uni{} ˨} & low tone& no\underline{!}\\
1 & {\uni{} ǀ} & dental click & \underline{tsk}tsk\\
8 & {\uni{} ǁ} & lateral click & winking \underline{click} \\
\end{tabular}
  \caption{ASCII alphabet used by Pyash,
  the letter's IPA equivalents, description and English pronunciation key.}\label{table:phonology}
\end{table}
 
First, an alphabet representing phonemes which are popular in human
languages was required, for this PHOIBLE\cite{phoible} was used.  Then WALS'\cite{WALS}
chapters on phoneme inventories was used to find what a common ratio of consonants to
vowels is, as well as common number of consonants and vowels, and picked the most
popular single phonemes which are reasonably distinct. Two tones were also
included to increase the number of words. Two clicks were included for 
temporary document specific words --- in place of acronyms. 
An ASCII letter for each IPA phoneme was also selected
(Table~\ref{table:phonology}) to
make sure Pyash is web compatible. 

Second, a morphology of how the phonemes are put together to make
words was required. For this phonotactics of the sonority scale\cite{sonority} was used,
paired with the WALS\cite{WALS} chapter on syllable structure. 

\begin{table}
  \begin{tabular}{llll}
    Code & ASCII Example & IPA & Name \\
    \midrule{}
    \textbf{CV}  &  ka & /{\uni{}kä}/ &  short grammar word \\
    \textbf{CSVH} & kyah & /{\uni{}kjäʰ}/ & long grammar word \\
    \textbf{HCVC} & hkap & /{\uni{}ʰkäp}/ & short root word \\
    \textbf{CSVC} & kyap & /{\uni{}kjäp}/ & long root word \\
  \end{tabular}
  \begin{description}
    \item[H] /{\uni{}ʰ}/ aspiration or spectrographically an unvoiced vowel. 
    \item[C] a consonant.
    \item[S] a consonant of higher sonority than the preceding one.
    \item[V] a vowel (highest sonority).
  \end{description}
  \caption{Pyash word morphology.
  }\label{table:morphology}
\end{table}

The language was also made easily parsed even if there are no spaces or pauses 
between words. Each word is either two or four letters long. The two letter
words start with a consonant and end with a vowel, and the four letter
ones start with two consonants and end with a consonant (Table~\ref{table:morphology}). 

The valid words were generated with several alphabets, and a script was made to
assign words based on the phonemes in the source languages weighed 
by their representative native speaking populations. The highest frequency 
words were assigned to the easier to pronounce and understand smaller alphabets.
And the more rare words were assigned to the more difficult extended alphabets ---
with voice contrast and-or tones for instance.

\subsection{Instruction Set Architecture}\label{ISA}
For complete vertical integration the language has to boil down to machine level
instruction, or an instruction set architecture. 
The JVM bytecode is an example of a different language which can also be
implemented as an instruction set architecture\cite{JOP}.

Understanding that the future of computing is going towards parallelism 
much research into how to make the language as parallel-friendly as possible was
done. 
In particular the Heads and Tails ISA\cite{headsTails} was found to be quite
inspiring. 

Each Pyash word fits in sixteen bits (a uint16\_t). There are four
word types and one quote type which are encoded. The quote type allows for
including literals. 


\begin{table}
  \begin{enumerate}
    \item \textbf{Pyash English}
    do say the quoted'word'hey world'word'quoted. 
  \item \textbf{Pyash} zi.wo.hwacwu.wo.zika hsactu  
  \item \textbf{Codelet}
   0051 291D E928 28BE  245E E948 295E 0000  0000 0000 0000 0000  0000 0000 0000 0000
 \item \textbf{Codelet Explained}
   (0051 index) (291D quoting two words) (E928 28BE hwac wu)  (245E ka
      accusative-case) (E948 hsac say) (295E tu deonitic-mood) 0000  0000 0000 0000 0000  0000 0000 0000 0000
    \item \textbf{C}
   \begin{lstlisting} 
   wotyutdokahsac(_("hwacwu"));
   \end{lstlisting} 
 \item \textbf{Output with en\_US locale} hey world \\
  \textbf{Output with ru locale} эй мир
  \end{enumerate}
\caption{A codelet encoding example.
  Note: Controlled natural language input and output was implemented in the 
  Javascript version\cite{speljs}, and hasn't yet been fully ported to C.
  }\label{table:codelet}
\end{table}

For parallelism sentences are encoded into codelets\cite{codelet}, which are comprised of one
or more vectors of sixteen, sixteen bit values. The first sixteen bit value of a
vector is the index for the vector, marking the location of grammatical cases
and moods (ends of noun and verb phrases). 


\begin{table}
  \begin{description}
    \item[Pyash] mina ryopyi syutka kwinli
    \item[Gloss] me NOM robot DAT liberty ACC giving REAL
    \item[Pyash English] I be giving the liberty to robot.
  \end{description}
  \caption{Example of formal translation}\label{table:translation}
\end{table}

This encoding can then be translated to any supported human language
(Table~\ref{table:translation}). In terms
of compiling to a programming language, it compiles to OpenCL C. There is also a
design\cite{vmOnOpenCL} for making a code-parallel virtual machine, that can process linear code
on GPU's using Pyash ISA.\@

The encoding could also be used for storage of information, similar to a
database, as well as for knowledge management, similar to how human languages
are used for storing information.

\subsection{Parser or Encoder}

The parser is probably of some interest due to its refined simplicity. 
It is a hand coded, single pass type, 
modeled on how a human would parse text. There are no parse trees or any such
complexities.

First the parser checks if a word is a valid Pyash word, if so, then checks if it is a
grammatical-case word,
 a grammatical-mood word or a quote word, if not then simply adds it to the
 codelet.

 If it is a quote word then acts
 accordingly either upon the literals ahead or the words behind,
 adding what is necessary to the codelet,
 and adjusting the codelet and text index pointer to just after the quote. 

 If it is a grammatical-case word, then in addition to adding the word to the codelet, 
 also marks it on the index. 

 If it is a grammatical-mood word then does as with the grammatical-case word
 but also ends the codelet. With the exception of the conditional mood, which is
 treated the same as a grammatical-case for encoding. 

 For reading and writing to the codelet there is a function, which manages which
 vector is being added to. If the addition over-runs one vector, then it's index
 is inverted, and the next vector receives the additions. This way when reading
 indexes, it is known if it is the end of the codelet based on the first bit of
 the index --- if it is a one then it is the final vector. 

This simple parser/encoder could parse/encode sentences in parallel, and should be
adaptable for parsing spoken streams of phonemes. A more complicated version of
the parser/encoder will be necessary once support is added for subordinate
clauses, since they would have to be broken up into multiple codelets for the
encoding. 

%\subsection{Translation to C}
%
%While the translation to C is not yet complete, can give a brief overview about
%the part that is operational. 
%
%First the last index point of the codelet is checked for the grammatical mood.
%If it is deontic, then the phrases are translated to function calls.
%


\section{Discussion}
%% results
%  After years of research, analysis, data-mining and prototypes have come up
%  with a language that not only allows for human-computer discourse and
%  programming, but surprisingly can also be usable as a highly formal 
%  interlingua between the majority of human languages. As of this writing, 
%  beta-testing seems to be just a short time away. 
Various variations of the language have been worked on since 2007. 
The first implementation\cite{rpoku} was in Haskell and second was in 
 Java\cite{rpokuJava}, both were recursive parsers.

The third implementation\cite{spel} followed the Jones Forth\cite{jonesForth} model, 
hoping to bootstrap something small and scaleable, so Intel
assembly was used for a few years and succeeded in making a basic interpreter.  

The fourth attempt\cite{speljs} was in nodejs Javascript, since by that time it
was realized that the language could be used for translation, and 
something portable was desired which could written quickly --- the antithesis of
assembly. A translator was made and a basic compiler to Javascript, but was severely limited
by a hand picked vocabulary, so an automated vocabulary (\ref{vocabulary}) was made.
While it was being made, it was realized that the object oriented 
implementation in Javascript was difficult or impossible to make parallel,
combined with its plain text encoding led to it running extremely slow. So the
Javascript translator and compiler was abandoned, but the automated vocabulary
was kept. 

The fifth and current attempt\cite{pyac} it was motivated by the realization that something fast,
scaleable and future-friendly was needed, so a parallelizeable ISA (\ref{ISA})
was designed
and the implementation was done in OpenCL C. As of this writing (May 2017) it 
compiles hello world, does variable assignment, for loops, and 
function declarations are being implemented.

\subsection{Vertical Integration}

While the main focus of the current implementations has been computer
programming languages and related documentation. The language can be used to
cover the areas of other software language types as well.

\subsubsection{Database Languages}
\begin{table}
  \begin{tabular}{lll}
    SQL & Pyash & Pyash English \\
    \midrule{}
    CREATE  & tlip & establish \\
    SELECT & kcot & gather \\
    UPDATE & draf & modernize \\
    DELETE & dlas & delete \\
    INSERT & hquk & inject \\
    FROM & pwih & from \\
    WHERE & te & at \\
    INTO & twih & into \\
  \end{tabular}
  \caption{Sampling of SQL keywords and their Pyash
  equivalents}\label{table:SQL}
\end{table}

For example SQL database access and creation language, can easily fit as a
subset of Pyash, with some slight vocabulary changes (Table~\ref{table:SQL}). 
Due to this rather fortunate grammatical-case design of SQL it should 
be possible to translate from SQL to Pyash and vice-versa --- 
whereas with most placement based
parameter family of languages it is a non-trivial process.

\subsubsection{Ontology Languages}
For knowledge representation or ontology languages, the databases could simply
be made of Pyash codelets. They could be rapidly queried in parallel on GPU for 
any particular piece of information. They could be translated to and from  human language,
for sharing gathered knowledge with humans, or acquiring knowledge from humans. 

Even a few people having a conversation, such as at a meeting could generate
programs and-or machine knowledge if they were speaking with enough formality to
be Pyash accessible.

Pyash accessibility is currently rather low, having a rather strict grammar. But
with machine learning algorithms to help with converting natural language speech
into Pyash controlled natural languages the amount of machine accessible 
knowledge that could be harvest from the spoken and written word should
dramatically increase.

\subsubsection{Modeling Languages}
Considering that Gellish is a modeling language, and that Pyash has a much more
developed grammar, it should be fairly straightforward to adapt Pyash to be a
universal modeling language. 

For visual people, graphics could be generated from Pyash descriptions. So in
the hypothetical scenario of some people talking in a meeting,  the computer
could be projecting the model of what is described on the screen. Or running and
showing simulations to see the potential outcomes of various policy or program 
changes. 

\subsubsection{Domain Specific Programming Languages}
The majority of domain specific languages seem to have placement based
parameters. This means that reading the API is likely necessary to understanding
how to use any functions.  Thus, unless the API is written in Pyash or some other
machine-accessible format, translating to and particularly from those languages
to Pyash is non-trivial. 

Translating to those languages is easier, as a human can read the API and make
an appropriate Pyash side function to access it.  However if someone adds a new
function to that other language, without following something like the Pyash
function naming convention, then it will be nearly impossibly to translate to
Pyash without reading it's corresponding API and-or analyzing it's code. 

Possibly when machine learning and AI gets sufficiently sophisticated it will be
able to do those translations, but that is quite possibly decades away. 

For now it makes sense to limit official Pyash programming development to 
compiling to popular C libraries, and also making native libraries.

\subsubsection{Communications Protocols and Serialization formats}
There are a wide range of communication protocols, all serving their own niches.
For example, HTTP, SMTP, and IRC.\@  

With the advent of XML there was an increase of protocol creation, for example
XMPP, SOAP and XML-RPC.\@ However since
XML doesn't have a root vocabulary most of these different protocols have
different naming conventions and so are not easily inter-operable. 

XML is also rather bulky, so in certain areas, such as configuration and data
storage, more compact alternative such as JSON, Lua and YAML have gained.  
Though like XML, they lack a root vocabulary. 

Pyash does have a root vocabulary so it is fairly straightforward to use as a
communication protocol. Having the root vocabulary could encourage people to
extend the language rather than make entirely new protocols. The binary 
encoding of Pyash, which can store various types including binary data, 
is both compact and can be decoded into a human readable format in a variety of
human languages. 

In terms of usage of space, Pyash is likely to be more bulky than any of the 
early terse ones like HTTP, but will typically use less space than XML, 
approaching JSON or YAML --- depending on the length of names used. 

The goal of using Pyash for protocols is making it easier to collect, 
consume and process large amounts of data.
Especially now that many of us have more storage and processing power than we know what to
do with. For example, may people have powerful GPU's in their computers, which
most of the time sit relatively idle. 

\begin{table}
  \begin{description}
    \item[Error message]  encoding:570:text\_encoding debug text
    \item[Pyash English] from encoding file at num five seven zero line in text encoding
      cereomony the debug text be emitting.
    \item[Pyash] kfinhfaspwih hfakhsipzrondo lyinlwoh htetkfinsricnwih
      dyekhtetka mwa7nli
  \end{description}
  \caption{Demonstrating how error messages might be conveyed more meaningfully
  using Pyash. }\label{errorMessage}
\end{table}

Since the error reporting was mentioned earlier, here is an example
(Table~\ref{errorMessage}). Though the Pyash versions are longer, 
they are more portable, and non-English speaking people can help debug the 
program, as the Pyash could be translated to an approximation of their 
native language. 

Additionally a variety of protocols could be translated into Pyash, not
necessarily so they would be faster, but to make it easier for an AI or AGI to
understand and communicate using them. 

\subsubsection{Markup Languages}
LaTeX, HTML and Markdown are some of the most popular markup languages on the
internet today. Of course they are mostly for formatting, and do not include a
vocabulary for the content.

However for writing modern documents, it is often important to have chapters,
sections and subsections.   Spoken speech has an (arguable) analog of bold 
and italics, via the focus and topic of the sentence --- which is already a part
of Pyash grammar. However spoken language
generally doesn't have long enough monologues for people to even mark their
spoken paragraphs.

The grammar of Pyash could be extended enough to allow for such mark up. An
example would be to make a grammar word for paragraph, module (section), and
frame (chapter). 

Pyash as a markup language would be particularly useful in using Pyash for writing
international content, such as stories, news articles or even legislation. 

\section{Conclusion and Further Work}
%% conclusion
%  In conclusion, a single language to unite all languages is viable, and has 
%  been implemented. It can be used as an interlingua between most, possibly all
%  human languages, and may be usable for all domains software languages are
%  typically used in, while maintaining the same grammar and vocabulary. 

A software language based on the fundamentals of human language that is usable
for human communication and computer programming is certainly viable and 
implementable, as it has been done. 

Translating all or most human languages, or at least controlled
variants of them does on the surface appear viable. Though further research
would have to be done to see what level of conjugation is comfortable for, and
how long it would take for native language speakers to adapt to the controlled
variants. 

Translating everything between software languages is unfortunately not
 viable due to the much smaller scope of them, as they can't be used for 
 human communication. Though existing codebase can be used via foreign function
 interface. 

Complete vertical integration of everything that a computer might need to do
seems to be viable, though further work would need to happen to prove it. 

This implementation of the language seems to be satisfactory. 
Language adoption is a major hurdle, which motivates this article.
Pyash is being used to write an automated programmer to more quickly write the
standard libraries, and general intelligence operating system to follow.

%\section{Introduction}

The \textit{proceedings} are the records of a conference\footnote{This
  is a footnote}.  ACM seeks
to give these conference by-products a uniform, high-quality
appearance.  To do this, ACM has some rigid requirements for the
format of the proceedings documents: there is a specified format
(balanced double columns), a specified set of fonts (Arial or
Helvetica and Times Roman) in certain specified sizes, a specified
live area, centered on the page, specified size of margins, specified
column width and gutter size.

\section{The Body of The Paper}
Typically, the body of a paper is organized into a hierarchical
structure, with numbered or unnumbered headings for sections,
subsections, sub-subsections, and even smaller sections.  The command
\texttt{{\char'134}section} that precedes this paragraph is part of
such a hierarchy.\footnote{This is a footnote.} \LaTeX\ handles the
numbering and placement of these headings for you, when you use the
appropriate heading commands around the titles of the headings.  If
you want a sub-subsection or smaller part to be unnumbered in your
output, simply append an asterisk to the command name.  Examples of
both numbered and unnumbered headings will appear throughout the
balance of this sample document.

Because the entire article is contained in the \textbf{document}
environment, you can indicate the start of a new paragraph with a
blank line in your input file; that is why this sentence forms a
separate paragraph.

\subsection{Type Changes and {\itshape Special} Characters}

We have already seen several typeface changes in this sample.  You can
indicate italicized words or phrases in your text with the command
\texttt{{\char'134}textit}; emboldening with the command
\texttt{{\char'134}textbf} and typewriter-style (for instance, for
computer code) with \texttt{{\char'134}texttt}.  But remember, you do
not have to indicate typestyle changes when such changes are part of
the \textit{structural} elements of your article; for instance, the
heading of this subsection will be in a sans serif\footnote{Another
  footnote, here.  Let's make this a rather short one to see how it
  looks.} typeface, but that is handled by the document class file.
Take care with the use of\footnote{A third, and last, footnote.}  the
curly braces in typeface changes; they mark the beginning and end of
the text that is to be in the different typeface.

You can use whatever symbols, accented characters, or non-English
characters you need anywhere in your document; you can find a complete
list of what is available in the \textit{\LaTeX\ User's Guide}
\cite{Lamport:LaTeX}.

\subsection{Math Equations}
You may want to display math equations in three distinct styles:
inline, numbered or non-numbered display.  Each of
the three are discussed in the next sections.

\subsubsection{Inline (In-text) Equations}
A formula that appears in the running text is called an
inline or in-text formula.  It is produced by the
\textbf{math} environment, which can be
invoked with the usual \texttt{{\char'134}begin\,\ldots{\char'134}end}
construction or with the short form \texttt{\$\,\ldots\$}. You
can use any of the symbols and structures,
from $\alpha$ to $\omega$, available in
\LaTeX~\cite{Lamport:LaTeX}; this section will simply show a
few examples of in-text equations in context. Notice how
this equation:
\begin{math}
  \lim_{n\rightarrow \infty}x=0
\end{math},
set here in in-line math style, looks slightly different when
set in display style.  (See next section).

\subsubsection{Display Equations}
A numbered display equation---one set off by vertical space from the
text and centered horizontally---is produced by the \textbf{equation}
environment. An unnumbered display equation is produced by the
\textbf{displaymath} environment.

Again, in either environment, you can use any of the symbols
and structures available in \LaTeX\@; this section will just
give a couple of examples of display equations in context.
First, consider the equation, shown as an inline equation above:
\begin{equation}
  \lim_{n\rightarrow \infty}x=0
\end{equation}
Notice how it is formatted somewhat differently in
the \textbf{displaymath}
environment.  Now, we'll enter an unnumbered equation:
\begin{displaymath}
  \sum_{i=0}^{\infty} x + 1
\end{displaymath}
and follow it with another numbered equation:
\begin{equation}
  \sum_{i=0}^{\infty}x_i=\int_{0}^{\pi+2} f
\end{equation}
just to demonstrate \LaTeX's able handling of numbering.

\subsection{Citations}
Citations to articles~\cite{bowman:reasoning,
clark:pct, braams:babel, herlihy:methodology},
conference proceedings~\cite{clark:pct} or maybe
books \cite{Lamport:LaTeX, salas:calculus} listed
in the Bibliography section of your
article will occur throughout the text of your article.
You should use BibTeX to automatically produce this bibliography;
you simply need to insert one of several citation commands with
a key of the item cited in the proper location in
the \texttt{.tex} file~\cite{Lamport:LaTeX}.
The key is a short reference you invent to uniquely
identify each work; in this sample document, the key is
the first author's surname and a
word from the title.  This identifying key is included
with each item in the \texttt{.bib} file for your article.

The details of the construction of the \texttt{.bib} file
are beyond the scope of this sample document, but more
information can be found in the \textit{Author's Guide},
and exhaustive details in the \textit{\LaTeX\ User's
Guide} by Lamport~\shortcite{Lamport:LaTeX}.


This article shows only the plainest form
of the citation command, using \texttt{{\char'134}cite}.

\subsection{Tables}
Because tables cannot be split across pages, the best
placement for them is typically the top of the page
nearest their initial cite.  To
ensure this proper ``floating'' placement of tables, use the
environment \textbf{table} to enclose the table's contents and
the table caption.  The contents of the table itself must go
in the \textbf{tabular} environment, to
be aligned properly in rows and columns, with the desired
horizontal and vertical rules.  Again, detailed instructions
on \textbf{tabular} material
are found in the \textit{\LaTeX\ User's Guide}.

Immediately following this sentence is the point at which
Table~\ref{tab:freq} is included in the input file; compare the
placement of the table here with the table in the printed
output of this document.

\begin{table}
  \caption{Frequency of Special Characters}
  \label{tab:freq}
  \begin{tabular}{ccl}
    \toprule
    Non-English or Math&Frequency&Comments\\
    \midrule
    \O & 1 in 1,000& For Swedish names\\
    $\pi$ & 1 in 5& Common in math\\
    \$ & 4 in 5 & Used in business\\
    $\Psi^2_1$ & 1 in 40,000& Unexplained usage\\
  \bottomrule
\end{tabular}
\end{table}

To set a wider table, which takes up the whole width of the page's
live area, use the environment \textbf{table*} to enclose the table's
contents and the table caption.  As with a single-column table, this
wide table will ``float'' to a location deemed more desirable.
Immediately following this sentence is the point at which
Table~\ref{tab:commands} is included in the input file; again, it is
instructive to compare the placement of the table here with the table
in the printed output of this document.


\begin{table*}
  \caption{Some Typical Commands}
  \label{tab:commands}
  \begin{tabular}{ccl}
    \toprule
    Command &A Number & Comments\\
    \midrule
    \texttt{{\char'134}author} & 100& Author \\
    \texttt{{\char'134}table}& 300 & For tables\\
    \texttt{{\char'134}table*}& 400& For wider tables\\
    \bottomrule
  \end{tabular}
\end{table*}
% end the environment with {table*}, NOTE not {table}!

It is strongly recommended to use the package booktabs~\cite{Fear05}
and follow its main principles of typography with respect to tables:
\begin{enumerate}
\item Never, ever use vertical rules.
\item Never use double rules.
\end{enumerate}
It is also a good idea not to overuse horizontal rules.


\subsection{Figures}

Like tables, figures cannot be split across pages; the best placement
for them is typically the top or the bottom of the page nearest their
initial cite.  To ensure this proper ``floating'' placement of
figures, use the environment \textbf{figure} to enclose the figure and
its caption.

This sample document contains examples of \texttt{.eps} files to be
displayable with \LaTeX.  If you work with pdf\LaTeX, use files in the
\texttt{.pdf} format.  Note that most modern \TeX\ systems will convert
\texttt{.eps} to \texttt{.pdf} for you on the fly.  More details on
each of these are found in the \textit{Author's Guide}.

\begin{figure}
\includegraphics{fly}
\caption{A sample black and white graphic.}
\end{figure}

\begin{figure}
\includegraphics[height=1in, width=1in]{fly}
\caption{A sample black and white graphic
that has been resized with the \texttt{includegraphics} command.}
\end{figure}


As was the case with tables, you may want a figure that spans two
columns.  To do this, and still to ensure proper ``floating''
placement of tables, use the environment \textbf{figure*} to enclose
the figure and its caption.  And don't forget to end the environment
with \textbf{figure*}, not \textbf{figure}!

\begin{figure*}
\includegraphics{flies}
\caption{A sample black and white graphic
that needs to span two columns of text.}
\end{figure*}


\begin{figure}
\includegraphics[height=1in, width=1in]{rosette}
\caption{A sample black and white graphic that has
been resized with the \texttt{includegraphics} command.}
\end{figure}

\subsection{Theorem-like Constructs}

Other common constructs that may occur in your article are the forms
for logical constructs like theorems, axioms, corollaries and proofs.
ACM uses two types of these constructs:  theorem-like and
definition-like.

Here is a theorem:
\begin{theorem}
  Let $f$ be continuous on $[a,b]$.  If $G$ is
  an antiderivative for $f$ on $[a,b]$, then
  \begin{displaymath}
    \int^b_af(t)\,dt = G(b) - G(a).
  \end{displaymath}
\end{theorem}

Here is a definition:
\begin{definition}
  If $z$ is irrational, then by $e^z$ we mean the
  unique number that has
  logarithm $z$:
  \begin{displaymath}
    \log e^z = z.
  \end{displaymath}
\end{definition}

The pre-defined theorem-like constructs are \textbf{theorem},
\textbf{conjecture}, \textbf{proposition}, \textbf{lemma} and
\textbf{corollary}.  The pre-defined de\-fi\-ni\-ti\-on-like constructs are
\textbf{example} and \textbf{definition}.  You can add your own
constructs using the \textsl{amsthm} interface~\cite{Amsthm15}.  The
styles used in the \verb|\theoremstyle| command are \textbf{acmplain}
and \textbf{acmdefinition}.

Another construct is \textbf{proof}, for example,

\begin{proof}
  Suppose on the contrary there exists a real number $L$ such that
  \begin{displaymath}
    \lim_{x\rightarrow\infty} \frac{f(x)}{g(x)} = L.
  \end{displaymath}
  Then
  \begin{displaymath}
    l=\lim_{x\rightarrow c} f(x)
    = \lim_{x\rightarrow c}
    \left[ g{x} \cdot \frac{f(x)}{g(x)} \right ]
    = \lim_{x\rightarrow c} g(x) \cdot \lim_{x\rightarrow c}
    \frac{f(x)}{g(x)} = 0\cdot L = 0,
  \end{displaymath}
  which contradicts our assumption that $l\neq 0$.
\end{proof}

\section{Conclusions}
This paragraph will end the body of this sample document.
Remember that you might still have Acknowledgments or
Appendices; brief samples of these
follow.  There is still the Bibliography to deal with; and
we will make a disclaimer about that here: with the exception
of the reference to the \LaTeX\ book, the citations in
this paper are to articles which have nothing to
do with the present subject and are used as
examples only.
%\end{document}  % This is where a 'short' article might terminate



\appendix
%Appendix A
\section{Headings in Appendices}
The rules about hierarchical headings discussed above for
the body of the article are different in the appendices.
In the \textbf{appendix} environment, the command
\textbf{section} is used to
indicate the start of each Appendix, with alphabetic order
designation (i.e., the first is A, the second B, etc.) and
a title (if you include one).  So, if you need
hierarchical structure
\textit{within} an Appendix, start with \textbf{subsection} as the
highest level. Here is an outline of the body of this
document in Appendix-appropriate form:
\subsection{Introduction}
\subsection{The Body of the Paper}
\subsubsection{Type Changes and  Special Characters}
\subsubsection{Math Equations}
\paragraph{Inline (In-text) Equations}
\paragraph{Display Equations}
\subsubsection{Citations}
\subsubsection{Tables}
\subsubsection{Figures}
\subsubsection{Theorem-like Constructs}
\subsubsection*{A Caveat for the \TeX\ Expert}
\subsection{Conclusions}
\subsection{References}
Generated by bibtex from your \texttt{.bib} file.  Run latex,
then bibtex, then latex twice (to resolve references)
to create the \texttt{.bbl} file.  Insert that \texttt{.bbl}
file into the \texttt{.tex} source file and comment out
the command \texttt{{\char'134}thebibliography}.
% This next section command marks the start of
% Appendix B, and does not continue the present hierarchy
\section{More Help for the Hardy}

Of course, reading the source code is always useful.  The file
\path{acmart.pdf} contains both the user guide and the commented
code.

\begin{acks}
  The authors would like to thank Dr. Yuhua Li for providing the
  matlab code of  the \textit{BEPS} method. 

  The authors would also like to thank the anonymous referees for
  their valuable comments and helpful suggestions. The work is
  supported by the \grantsponsor{GS501100001809}{National Natural
    Science Foundation of
    China}{http://dx.doi.org/10.13039/501100001809} under Grant
  No.:~\grantnum{GS501100001809}{61273304}
  and~\grantnum[http://www.nnsf.cn/youngscientsts]{GS501100001809}{Young
    Scientsts' Support Program}.

\end{acks}


\bibliographystyle{ACM-Reference-Format}
\bibliography{pyash} 

\end{document}

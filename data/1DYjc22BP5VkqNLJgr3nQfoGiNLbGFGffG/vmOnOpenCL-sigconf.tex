\documentclass[sigconf]{acmart}
\usepackage{bytefield}
\usepackage{hyperref}
\usepackage{tabulary}
\usepackage{listings}
\usepackage{booktabs} % For formal tables
\usepackage{tikz}
\usetikzlibrary{arrows,shapes,automata,petri,positioning}

\tikzset{place/.style={circle,
        thick,
        draw=blue!75,
        fill=blue!20,
        minimum size=6mm,
    },
    transitionH/.style={rectangle,
        thick,
        fill=black,
        minimum width=8mm,
        inner ysep=2pt
    },
    transitionV/.style={rectangle,
        thick,
        fill=black,
        minimum height=8mm,
        inner xsep=2pt
    }
}


% Copyright
%\setcopyright{none}
%\setcopyright{acmcopyright}
%\setcopyright{acmlicensed}
%\setcopyright{rightsretained}
%\setcopyright{usgov}
%\setcopyright{usgovmixed}
%\setcopyright{cagov}
%\setcopyright{cagovmixed}


% DOI
%\acmDOI{10.475/123_4}

% ISBN
%\acmISBN{123-4567-24-567/08/06}

%Conference
\acmConference[IWOCL'17]{IWOCL OpenCL conference}{May 2017}{Toronto, Ontario,
Canada} 
\acmYear{2017}
\copyrightyear{2017}

%\acmPrice{15.00}


\begin{document}
\title[VM on OpenCL]{Expanding domain of algorithms for GPGPU with code
parallelism}
\titlenote{Expanding domain of algorithms that can use GPU with a codelet
bytecode interpreter running as an OpenCL kernel}
%\subtitle{Extended Abstract}
%\subtitlenote{The full version of the author's guide is available as
%  \texttt{acmart.pdf} document}


\author{Htaf (Logan) Dwes}
%\authornote{}
%\orcid{1234-5678-9012}
\affiliation{%
  \institution{LiberIT Liberty Information Technology Services}
  %\streetaddress{P.O. Box 1212}
  \city{Owen Sound} 
  \state{Ontario} 
  \postcode{N4K 4R1}
}
\email{logan@liberit.ca}

% The default list of authors is too long for headers}
\renewcommand{\shortauthors}{Logan Dwes}


\begin{abstract}%motivation
Motivated by the desire to maximize use of cheap GPU processing power by
increasing the kinds of algorithms that can profit from it.
% problem statement
Currently only data parallel algorithms can profit from running their code on 
GPU.\@
% approach
One way is by making a programming language whose code can be treated as data 
and streamed through an OpenCL kernel. 
% results
Herein is the design of a programming language that allows using code as data
that can be streamed through an OpenCL kernel. 
% conclusion
This way not only data parallel
algorithms can reap the benefits of GPU, but also any referentially transparent
code can. 
\end{abstract}

%
% The code below should be generated by the tool at
% http://dl.acm.org/ccs.cfm
% Please copy and paste the code instead of the example below. 
%

\begin{CCSXML}
<ccs2012>
<concept>
<concept_id>10011007.10011006.10011041.10010943</concept_id>
<concept_desc>Software and its engineering~Interpreters</concept_desc>
<concept_significance>300</concept_significance>
</concept>
</ccs2012>
\end{CCSXML}

\ccsdesc[300]{Software and its engineering~Interpreters}

% We no longer use \terms command
%\terms{Theory}

\keywords{OpenCL, virtual-machine, interpreter}


%\maketitle

\maketitle{}
\section{Introduction}

The plan is to make a next generation programming language for programming AGI\@.
Constraints include using human grammar (linguistic universals), being
compatible with genetic programming and maximize GPU usage, which is some of the
cheapest and most underutilized processing power we have available.

In this paper only focusing on the maximizing GPU usage via the virtual machine
which can run the intermediary representation.  The intermediary code can also
be compiled to C (host code), and OpenCL C (kernels), particularly for more
traditional data-parallel applications. 

But for the many processes which are not data-parallel, and instead have long
computations such as compiling \LaTeX{} files, those can be run through a virtual
machine sitting implemented as an OpenCL kernel.   

In fact the programming language like functional programming languages,
encourages to keep all the input and output in the main function or monad,
wheras the ones which are called are all referentially transparent.  

This way can load-balance an application over as many cores as are available,
including GPUs. 

\section{Previous Works}

There are many works talking about getting OpenCL working inside a virtual
machine, such as
KVM\cite{SPE:SPE2166}\cite{Gupta:2009:GGV:1519138.1519141}\cite{ratering2011accelerating}, but that is completely different from having a VM running
ontop of OpenCL.\@

One that sounds similar is ``OpenCL for Interpreter Implementation''\cite{OpenCLInterpret}
Though it compiles virtual machine bytecode to OpenCL kernels on CPU, so
all it really shows is that compiled code runs faster on CPU than interpreted code on CPU.\@

Another similar one is ``Parallel Programming in Actor-Based Applications via OpenCL''
\cite{Harvey:2015:PPA:2814576.2814732}, which talks about implementing actors 
while using OpenCL\@. Though actors are quite different from bytecode
interpreters. And to use this would require translating code to an actor model,
which may be difficult and not practical for many applications. 

Instead I looked at highly parallel instruction set architectures, in particular 
the architecture used by the intermediary language is partially inspired by VLIW heads-or-tails architecture 
\cite{Pan:2001:HTV:502217.502244}

\section{Operating Template}

A codelet is a self-contained code module, the linguistic parallel is an
independent clause familiarly known as a sentence. 

Each codelet or independent-clause has several phrases for input and output,
which are indexed by it.

The bytecodes or words that make up the phrases\ref{phraseOverview} are all equal width.
\begin{table}
\begin{bytefield}[endianness=little, bitwidth=0.0625\linewidth]{16}
  \bitheader{0,1,2,3,4,8,12,15}\\
\bitbox{1}{I} &
\bitbox{1}{Q}&\bitbox{1}{C}&\bitbox{1}{P} &
   \bitbox{1}{Q} &   \bitbox{1}{C} & \bitbox{1}{C} 
 & \bitbox{1}{P} \bitbox{1}{Q} &   \bitbox{1}{C} & \bitbox{1}{C} 
&   \bitbox{1}{C} & \bitbox{1}{C} & \bitbox{1}{P} & \bitbox{1}{V} 
& \bitbox{1}{M} \\

\end{bytefield}
\caption{Codelet layout, composed of one ushort16, 
a 16bit phrase, a 32bit phrase, and 64bit phrase are demonstrated.}
\label{phraseOverview}
\begin{tabulary}{0.5\textwidth}{LL} 
  I & Index \\
  Q & Quote denote \\
  C & Content or quoted value, number of ushorts it is composed of varies depending on bit length of value \\
  P & Phrase end word or grammatical-case\\
  V & Verb or command that operates on the phrases\\
  M & Mood word, or grammatical mood (end of sentence)\\
  U & Unassigned words after end of sentence \\
\end{tabulary}
\end{table}

In the Pyash implementation the codelets are each a ushort16 vector. If that is
not enough to contain for instance a double16 constant, then the index
(contained in the first ushort) indicates
that it is not the final ushort16 (see table-\ref{phraseOverview})


Each interpreting worker reads one of the ushort16s in the code array, if index 
starts with a partial then it skips to the next global id plus work group size
short16. Though before it does, at the end of each evaluation all the workers
synchronize to avoid race conditions.  


\begin{table}
\begin{bytefield}[endianness=little, bitwidth=0.0625\linewidth]{16}
  \bitheader{0,1,4,8,12,15}\\
  \bitbox{1}{c} &   \bitbox{1}{p} & \bitbox{1}{p} & \bitbox{1}{p} 
  & \bitbox{1}{p} & \bitbox{1}{p} & \bitbox{1}{p} & \bitbox{1}{p} 
  & \bitbox{1}{p} & \bitbox{1}{p} & \bitbox{1}{p} & \bitbox{1}{p} 
  & \bitbox{1}{p} & \bitbox{1}{p} & \bitbox{1}{p} & \bitbox{1}{p} \\
  \bitbox{1}{1} &   \bitbox{1}{0} & \bitbox{1}{0} & \bitbox{1}{1} 
  & \bitbox{1}{0} & \bitbox{1}{0} & \bitbox{1}{0} & \bitbox{1}{1} 
  & \bitbox{1}{0} & \bitbox{1}{0} & \bitbox{1}{0} & \bitbox{1}{0} 
  & \bitbox{1}{0} & \bitbox{1}{1} & \bitbox{1}{0} & \bitbox{1}{1} \\

\end{bytefield}
\caption{Index Overview}
\label{indexOverview}
\begin{tabulary}{0.5\textwidth}{LL} 
  c & Completion bit indicator, if equals 0 then ushort16 is only part of codelet \\
  p & Phrase or mood bit indicator, if is equal to completion bit, then a phrase
word or mood word is here.\\
\end{tabulary}
\end{table}

If it is marked as final, it checks preceding indexes to get any extra short16's
that make up the codelet, then evaluates it. 

\subsection{Memory Template}
The program code is loaded into constant memory. The working memory is in an
globally indexed local memory heap, and output is to global memory.
%(see figure~\ref{memoryTemplate}).

Each variable has a reference number in the referential phrase.
The global index indicates if it has been set, and it's location in the local
memory heap.  The worker waits until all inputs are set before evaluating the 
codelet. 

%\begin{figure}
%\begin{tikzpicture}[node distance=0.5cm and 1cm,>=stealth',bend angle=45,auto]
%    \node [place,label=above:$C$] (p1) {};
%    \node [place,label=above:$D$] (d) [below= of p1] {};
%    \node [transitionV,label=above:$W_1$] (t1) [right= of p1] {}
%        edge[pre]   (p1)
%        edge[pre]   (d);
%    \node [transitionV,label=above:$W_n$] (wn) [right= of d] {}
%        edge[pre]   (p1)
%        edge[pre]   (d);
%    \node [place,tokens=1,label=above:$I$] (p2) [above right=of t1] {}
%        edge[pre,out=-135,in=30,looseness=1,overlay]   (t1)
%        edge[post,out=180,in=60,looseness=1,overlay]   (t1);
%    \node [place,tokens=2,label=above:$P_3$] (p3) [below right=of t1] {}
%        edge[pre]   (t1);
%    \node [transitionV,label=above:$T_2$] (t2) [above right=of p3] {}
%        edge[pre]   (p2)
%        edge[pre]   (p3)
%        edge[post,out=-110,in=-50,looseness=2,overlay]  (p1);
%    \node [place,tokens=1, label=above:$P_4$] (p4) [above right=of t2] {}
%        edge[pre]   (t2);
%\end{tikzpicture}
%\caption{Memory Template}
%\label{memoryTemplate}
%\begin{tabular}{ll}
%  C & Code in constant memory \\
%  D & input Data in constant memory \\
%  V & Variable index in global memory \\
%  W & Worker \\
%  I & Intermediate values processed in local memory\\
%  P & Produced data in global memory \\
%\end{tabular}
%\end{figure}


\subsection{Control Flow}
Control flow is managed through variables checked by codelet conditionals. 

For example a comparison codelet sets an output variable, 
all the codelets whose execution requires the knowledge of that comparisons
value check that it is set and that it passes their internal conditional before
evaluating their codelet.\ 

\begin{lstlisting}
result = 1 > 2;
if (result == true) expression1();
if (result == false) expression2();
\end{lstlisting}



All workers check to see if the program is still running, to avoid hangs.\ 

\begin{lstlisting}
if (running == TRUE) result = 1 > 2;
if (running == TRUE && result == TRUE) 
  expression1();
if (running == TRUE && result == FALSE) 
  expression2();
\end{lstlisting}
See 


\begin{table}
\begin{bytefield}[endianness=little, bitwidth=0.0625\linewidth]{16}
  \bitheader{0,1,2,3,4,8,12,15}\\
\bitbox{1}{I} &
\bitbox{1}{Q}&\bitbox{1}{C}&\bitbox{1}{P} & \bitbox{1}{Q}
&\bitbox{1}{C}&\bitbox{1}{P} & \bitbox{1}{V}&\bitbox{1}{M} & 
\bitbox{1}{Q}&\bitbox{1}{C}&\bitbox{1}{P} & \bitbox{1}{Q}
&\bitbox{1}{C}&\bitbox{1}{P} & \bitbox{1}{V} \\
\bitbox{1}{I} &\bitbox{1}{M} & 

\bitbox{1}{Q}&\bitbox{1}{C}&\bitbox{1}{P} & \bitbox{1}{Q} &   
\bitbox{1}{C} & \bitbox{1}{C} & \bitbox{1}{P} 
\bitbox{1}{Q} 
&   \bitbox{1}{C}  \bitbox{1}{C} &\bitbox{1}{P} & \bitbox{1}{V} 
& \bitbox{1}{M} & \bitbox{1}{U} \\

\end{bytefield}
\caption{Multi ushort16 Codelet layout, includes two conditional clauses,
a 16bit phrase, a 32bit phrase, and 64bit phrase, are demonstrated.}
\label{multiPhraseOverview}
\begin{tabulary}{0.5\textwidth}{LL} 
  I & Index \\
  Q & Quote denote \\
  C & Content or quoted value, number of ushorts it is composed of varies depending on bit length of value \\
  P & Phrase end word or grammatical-case\\
  V & Verb or command that operates on the phrases\\
  M & Mood word, or grammatical mood (end of clause)\\
  U & Unassigned words after end of sentence \\
\end{tabulary}
\end{table}


If there are more layers of conditionals, then worker has to check them all.\ 

\begin{lstlisting}
if (running == TRUE && result == TRUE) 
  result2 = 4 > 2;
if (running == TRUE && result == TRUE && 
    result2 == TRUE) 
  expression3();
\end{lstlisting}
Can see an example of the layouf of a codelet that has multiple conditionals in
figure~\ref{multiPhraseOverview}


\subsubsection{Loops}
All loops that don't break or return should simply be unrolled. 

Otherwise there are two options for implementation, 

The prefered option is simply to only have static length for loops, 
and always unroll them, having a variable to check if it has been broken.  

Of course some for loop lengths are set at runtime
for those can have a special looping variable which each codelet checks to see
if it should continue. It can check after the synchronization point to know if
it should jump to the next codelet or continue evaluating this one. 

\subsubsection{Jumps}
Jumps may be necessary for some pieces of low-probability code. If there is a
branch to a chunk of code, the branching worker can set a jump variable with the
location (in constant memory) and length of the code.  Then other workers would
jump to evaluating their corresponding part of that code, or continue on the
main code if they don't fit. 


\section{Speculation}
It may not lead to significant performance gain, as it is interpreted rather
than compiled, though it does inherently support superscalar execution,
out-of-order execution, and speculative execution simply because all the
codelets are executed in parallel, so
may be quite fast, especially if implemented as a core architecure.

\section{Conclusion and Further Work}

This programming language may lead to increased usage of GPU's for a greater
diversity of tasks. 
As of this writing I only have a basic prototype of the language, though with
more time and effort it can become fully functional. 


%\section{Introduction}

The \textit{proceedings} are the records of a conference\footnote{This
  is a footnote}.  ACM seeks
to give these conference by-products a uniform, high-quality
appearance.  To do this, ACM has some rigid requirements for the
format of the proceedings documents: there is a specified format
(balanced double columns), a specified set of fonts (Arial or
Helvetica and Times Roman) in certain specified sizes, a specified
live area, centered on the page, specified size of margins, specified
column width and gutter size.

\section{The Body of The Paper}
Typically, the body of a paper is organized into a hierarchical
structure, with numbered or unnumbered headings for sections,
subsections, sub-subsections, and even smaller sections.  The command
\texttt{{\char'134}section} that precedes this paragraph is part of
such a hierarchy.\footnote{This is a footnote.} \LaTeX\ handles the
numbering and placement of these headings for you, when you use the
appropriate heading commands around the titles of the headings.  If
you want a sub-subsection or smaller part to be unnumbered in your
output, simply append an asterisk to the command name.  Examples of
both numbered and unnumbered headings will appear throughout the
balance of this sample document.

Because the entire article is contained in the \textbf{document}
environment, you can indicate the start of a new paragraph with a
blank line in your input file; that is why this sentence forms a
separate paragraph.

\subsection{Type Changes and {\itshape Special} Characters}

We have already seen several typeface changes in this sample.  You can
indicate italicized words or phrases in your text with the command
\texttt{{\char'134}textit}; emboldening with the command
\texttt{{\char'134}textbf} and typewriter-style (for instance, for
computer code) with \texttt{{\char'134}texttt}.  But remember, you do
not have to indicate typestyle changes when such changes are part of
the \textit{structural} elements of your article; for instance, the
heading of this subsection will be in a sans serif\footnote{Another
  footnote, here.  Let's make this a rather short one to see how it
  looks.} typeface, but that is handled by the document class file.
Take care with the use of\footnote{A third, and last, footnote.}  the
curly braces in typeface changes; they mark the beginning and end of
the text that is to be in the different typeface.

You can use whatever symbols, accented characters, or non-English
characters you need anywhere in your document; you can find a complete
list of what is available in the \textit{\LaTeX\ User's Guide}
\cite{Lamport:LaTeX}.

\subsection{Math Equations}
You may want to display math equations in three distinct styles:
inline, numbered or non-numbered display.  Each of
the three are discussed in the next sections.

\subsubsection{Inline (In-text) Equations}
A formula that appears in the running text is called an
inline or in-text formula.  It is produced by the
\textbf{math} environment, which can be
invoked with the usual \texttt{{\char'134}begin\,\ldots{\char'134}end}
construction or with the short form \texttt{\$\,\ldots\$}. You
can use any of the symbols and structures,
from $\alpha$ to $\omega$, available in
\LaTeX~\cite{Lamport:LaTeX}; this section will simply show a
few examples of in-text equations in context. Notice how
this equation:
\begin{math}
  \lim_{n\rightarrow \infty}x=0
\end{math},
set here in in-line math style, looks slightly different when
set in display style.  (See next section).

\subsubsection{Display Equations}
A numbered display equation---one set off by vertical space from the
text and centered horizontally---is produced by the \textbf{equation}
environment. An unnumbered display equation is produced by the
\textbf{displaymath} environment.

Again, in either environment, you can use any of the symbols
and structures available in \LaTeX\@; this section will just
give a couple of examples of display equations in context.
First, consider the equation, shown as an inline equation above:
\begin{equation}
  \lim_{n\rightarrow \infty}x=0
\end{equation}
Notice how it is formatted somewhat differently in
the \textbf{displaymath}
environment.  Now, we'll enter an unnumbered equation:
\begin{displaymath}
  \sum_{i=0}^{\infty} x + 1
\end{displaymath}
and follow it with another numbered equation:
\begin{equation}
  \sum_{i=0}^{\infty}x_i=\int_{0}^{\pi+2} f
\end{equation}
just to demonstrate \LaTeX's able handling of numbering.

\subsection{Citations}
Citations to articles~\cite{bowman:reasoning,
clark:pct, braams:babel, herlihy:methodology},
conference proceedings~\cite{clark:pct} or maybe
books \cite{Lamport:LaTeX, salas:calculus} listed
in the Bibliography section of your
article will occur throughout the text of your article.
You should use BibTeX to automatically produce this bibliography;
you simply need to insert one of several citation commands with
a key of the item cited in the proper location in
the \texttt{.tex} file~\cite{Lamport:LaTeX}.
The key is a short reference you invent to uniquely
identify each work; in this sample document, the key is
the first author's surname and a
word from the title.  This identifying key is included
with each item in the \texttt{.bib} file for your article.

The details of the construction of the \texttt{.bib} file
are beyond the scope of this sample document, but more
information can be found in the \textit{Author's Guide},
and exhaustive details in the \textit{\LaTeX\ User's
Guide} by Lamport~\shortcite{Lamport:LaTeX}.


This article shows only the plainest form
of the citation command, using \texttt{{\char'134}cite}.

\subsection{Tables}
Because tables cannot be split across pages, the best
placement for them is typically the top of the page
nearest their initial cite.  To
ensure this proper ``floating'' placement of tables, use the
environment \textbf{table} to enclose the table's contents and
the table caption.  The contents of the table itself must go
in the \textbf{tabular} environment, to
be aligned properly in rows and columns, with the desired
horizontal and vertical rules.  Again, detailed instructions
on \textbf{tabular} material
are found in the \textit{\LaTeX\ User's Guide}.

Immediately following this sentence is the point at which
Table~\ref{tab:freq} is included in the input file; compare the
placement of the table here with the table in the printed
output of this document.

\begin{table}
  \caption{Frequency of Special Characters}
  \label{tab:freq}
  \begin{tabular}{ccl}
    \toprule
    Non-English or Math&Frequency&Comments\\
    \midrule
    \O & 1 in 1,000& For Swedish names\\
    $\pi$ & 1 in 5& Common in math\\
    \$ & 4 in 5 & Used in business\\
    $\Psi^2_1$ & 1 in 40,000& Unexplained usage\\
  \bottomrule
\end{tabular}
\end{table}

To set a wider table, which takes up the whole width of the page's
live area, use the environment \textbf{table*} to enclose the table's
contents and the table caption.  As with a single-column table, this
wide table will ``float'' to a location deemed more desirable.
Immediately following this sentence is the point at which
Table~\ref{tab:commands} is included in the input file; again, it is
instructive to compare the placement of the table here with the table
in the printed output of this document.


\begin{table*}
  \caption{Some Typical Commands}
  \label{tab:commands}
  \begin{tabular}{ccl}
    \toprule
    Command &A Number & Comments\\
    \midrule
    \texttt{{\char'134}author} & 100& Author \\
    \texttt{{\char'134}table}& 300 & For tables\\
    \texttt{{\char'134}table*}& 400& For wider tables\\
    \bottomrule
  \end{tabular}
\end{table*}
% end the environment with {table*}, NOTE not {table}!

It is strongly recommended to use the package booktabs~\cite{Fear05}
and follow its main principles of typography with respect to tables:
\begin{enumerate}
\item Never, ever use vertical rules.
\item Never use double rules.
\end{enumerate}
It is also a good idea not to overuse horizontal rules.


\subsection{Figures}

Like tables, figures cannot be split across pages; the best placement
for them is typically the top or the bottom of the page nearest their
initial cite.  To ensure this proper ``floating'' placement of
figures, use the environment \textbf{figure} to enclose the figure and
its caption.

This sample document contains examples of \texttt{.eps} files to be
displayable with \LaTeX.  If you work with pdf\LaTeX, use files in the
\texttt{.pdf} format.  Note that most modern \TeX\ systems will convert
\texttt{.eps} to \texttt{.pdf} for you on the fly.  More details on
each of these are found in the \textit{Author's Guide}.

\begin{figure}
\includegraphics{fly}
\caption{A sample black and white graphic.}
\end{figure}

\begin{figure}
\includegraphics[height=1in, width=1in]{fly}
\caption{A sample black and white graphic
that has been resized with the \texttt{includegraphics} command.}
\end{figure}


As was the case with tables, you may want a figure that spans two
columns.  To do this, and still to ensure proper ``floating''
placement of tables, use the environment \textbf{figure*} to enclose
the figure and its caption.  And don't forget to end the environment
with \textbf{figure*}, not \textbf{figure}!

\begin{figure*}
\includegraphics{flies}
\caption{A sample black and white graphic
that needs to span two columns of text.}
\end{figure*}


\begin{figure}
\includegraphics[height=1in, width=1in]{rosette}
\caption{A sample black and white graphic that has
been resized with the \texttt{includegraphics} command.}
\end{figure}

\subsection{Theorem-like Constructs}

Other common constructs that may occur in your article are the forms
for logical constructs like theorems, axioms, corollaries and proofs.
ACM uses two types of these constructs:  theorem-like and
definition-like.

Here is a theorem:
\begin{theorem}
  Let $f$ be continuous on $[a,b]$.  If $G$ is
  an antiderivative for $f$ on $[a,b]$, then
  \begin{displaymath}
    \int^b_af(t)\,dt = G(b) - G(a).
  \end{displaymath}
\end{theorem}

Here is a definition:
\begin{definition}
  If $z$ is irrational, then by $e^z$ we mean the
  unique number that has
  logarithm $z$:
  \begin{displaymath}
    \log e^z = z.
  \end{displaymath}
\end{definition}

The pre-defined theorem-like constructs are \textbf{theorem},
\textbf{conjecture}, \textbf{proposition}, \textbf{lemma} and
\textbf{corollary}.  The pre-defined de\-fi\-ni\-ti\-on-like constructs are
\textbf{example} and \textbf{definition}.  You can add your own
constructs using the \textsl{amsthm} interface~\cite{Amsthm15}.  The
styles used in the \verb|\theoremstyle| command are \textbf{acmplain}
and \textbf{acmdefinition}.

Another construct is \textbf{proof}, for example,

\begin{proof}
  Suppose on the contrary there exists a real number $L$ such that
  \begin{displaymath}
    \lim_{x\rightarrow\infty} \frac{f(x)}{g(x)} = L.
  \end{displaymath}
  Then
  \begin{displaymath}
    l=\lim_{x\rightarrow c} f(x)
    = \lim_{x\rightarrow c}
    \left[ g{x} \cdot \frac{f(x)}{g(x)} \right ]
    = \lim_{x\rightarrow c} g(x) \cdot \lim_{x\rightarrow c}
    \frac{f(x)}{g(x)} = 0\cdot L = 0,
  \end{displaymath}
  which contradicts our assumption that $l\neq 0$.
\end{proof}

\section{Conclusions}
This paragraph will end the body of this sample document.
Remember that you might still have Acknowledgments or
Appendices; brief samples of these
follow.  There is still the Bibliography to deal with; and
we will make a disclaimer about that here: with the exception
of the reference to the \LaTeX\ book, the citations in
this paper are to articles which have nothing to
do with the present subject and are used as
examples only.
%\end{document}  % This is where a 'short' article might terminate



\appendix
%Appendix A
\section{Headings in Appendices}
The rules about hierarchical headings discussed above for
the body of the article are different in the appendices.
In the \textbf{appendix} environment, the command
\textbf{section} is used to
indicate the start of each Appendix, with alphabetic order
designation (i.e., the first is A, the second B, etc.) and
a title (if you include one).  So, if you need
hierarchical structure
\textit{within} an Appendix, start with \textbf{subsection} as the
highest level. Here is an outline of the body of this
document in Appendix-appropriate form:
\subsection{Introduction}
\subsection{The Body of the Paper}
\subsubsection{Type Changes and  Special Characters}
\subsubsection{Math Equations}
\paragraph{Inline (In-text) Equations}
\paragraph{Display Equations}
\subsubsection{Citations}
\subsubsection{Tables}
\subsubsection{Figures}
\subsubsection{Theorem-like Constructs}
\subsubsection*{A Caveat for the \TeX\ Expert}
\subsection{Conclusions}
\subsection{References}
Generated by bibtex from your \texttt{.bib} file.  Run latex,
then bibtex, then latex twice (to resolve references)
to create the \texttt{.bbl} file.  Insert that \texttt{.bbl}
file into the \texttt{.tex} source file and comment out
the command \texttt{{\char'134}thebibliography}.
% This next section command marks the start of
% Appendix B, and does not continue the present hierarchy
\section{More Help for the Hardy}

Of course, reading the source code is always useful.  The file
\path{acmart.pdf} contains both the user guide and the commented
code.

\begin{acks}
  The authors would like to thank Dr. Yuhua Li for providing the
  matlab code of  the \textit{BEPS} method. 

  The authors would also like to thank the anonymous referees for
  their valuable comments and helpful suggestions. The work is
  supported by the \grantsponsor{GS501100001809}{National Natural
    Science Foundation of
    China}{http://dx.doi.org/10.13039/501100001809} under Grant
  No.:~\grantnum{GS501100001809}{61273304}
  and~\grantnum[http://www.nnsf.cn/youngscientsts]{GS501100001809}{Young
    Scientsts' Support Program}.

\end{acks}


\bibliographystyle{ACM-Reference-Format}
\bibliography{sigproc} 

\end{document}
